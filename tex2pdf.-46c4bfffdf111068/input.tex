% Options for packages loaded elsewhere
\PassOptionsToPackage{unicode}{hyperref}
\PassOptionsToPackage{hyphens}{url}
%
\documentclass[
]{article}
\usepackage{amsmath,amssymb}
\usepackage{iftex}
\ifPDFTeX
  \usepackage[T1]{fontenc}
  \usepackage[utf8]{inputenc}
  \usepackage{textcomp} % provide euro and other symbols
\else % if luatex or xetex
  \usepackage{unicode-math} % this also loads fontspec
  \defaultfontfeatures{Scale=MatchLowercase}
  \defaultfontfeatures[\rmfamily]{Ligatures=TeX,Scale=1}
\fi
\usepackage{lmodern}
\ifPDFTeX\else
  % xetex/luatex font selection
\fi
% Use upquote if available, for straight quotes in verbatim environments
\IfFileExists{upquote.sty}{\usepackage{upquote}}{}
\IfFileExists{microtype.sty}{% use microtype if available
  \usepackage[]{microtype}
  \UseMicrotypeSet[protrusion]{basicmath} % disable protrusion for tt fonts
}{}
\makeatletter
\@ifundefined{KOMAClassName}{% if non-KOMA class
  \IfFileExists{parskip.sty}{%
    \usepackage{parskip}
  }{% else
    \setlength{\parindent}{0pt}
    \setlength{\parskip}{6pt plus 2pt minus 1pt}}
}{% if KOMA class
  \KOMAoptions{parskip=half}}
\makeatother
\usepackage{xcolor}
\setlength{\emergencystretch}{3em} % prevent overfull lines
\providecommand{\tightlist}{%
  \setlength{\itemsep}{0pt}\setlength{\parskip}{0pt}}
\setcounter{secnumdepth}{-\maxdimen} % remove section numbering
\ifLuaTeX
  \usepackage{selnolig}  % disable illegal ligatures
\fi
\IfFileExists{bookmark.sty}{\usepackage{bookmark}}{\usepackage{hyperref}}
\IfFileExists{xurl.sty}{\usepackage{xurl}}{} % add URL line breaks if available
\urlstyle{same}
\hypersetup{
  hidelinks,
  pdfcreator={LaTeX via pandoc}}

\author{}
\date{}

\begin{document}

{[}insérer tt les cm/td loupés{]} {[}{[}lire ces cm de mort{]}{]}

\begin{center}\rule{0.5\linewidth}{0.5pt}\end{center}

\hypertarget{le-guxe9nome-plastidial}{%
\subsection{Le génome plastidial}\label{le-guxe9nome-plastidial}}

\hypertarget{introduction}{%
\subsubsection{Introduction}\label{introduction}}

\hypertarget{i.-caractuxe9ristiques-guxe9nuxe9rales-du-guxe9nome-plastidial-transmission-structure-organisation}{%
\subsubsection{I. Caractéristiques générales du génome plastidial:
transmission, structure,
organisation}\label{i.-caractuxe9ristiques-guxe9nuxe9rales-du-guxe9nome-plastidial-transmission-structure-organisation}}

{[}insérer cm{]} \textgreater{} plastes d’origine procaryotiques dont a
conservé des caractéristiques procaryotypes

Génome circulaire (caractéristique de type procaryote) {[}{[}Pasted
image 20230202095432.png{]}{]} {[}{[}Pasted image
20230202095851.png{]}{]} \#\#\#\#\# Organisation des gènes plastidiaux :

\begin{itemize}
\tightlist
\item
  la majorité des gènes plastidiaux sont organisés en \textbf{opérons}

  \begin{itemize}
  \tightlist
  \item
    transcrits sous la forme d’\textbf{ARNm polycistroniques} (=type
    procarypte) {[}{[}Pasted image 20230202095815.png{]}{]}

    monocistron : ce qu’on trouverait dans un génome nucléaire (donc
    eucaryotique) ; la transcription de ces gènes pour laquelle l’ARNm
    produit correspond à la séquence du gène transcrite à partir d’un
    promoteur qui lui est spécifique (1 promoteur-1gène)

    polycistron : un ensemble de gènes seront transcrit ensemble sous la
    forme d’un ARN unique

    rbcl (code grande sous unité RubisCo) monocistroniques ; les
    monocistroniques ont souvent une importance forte pour la
    photosynthèse
  \end{itemize}
\item
  La majorité des gènes n’ont \textbf{pas d’introns} (=type procaryote)

  \begin{itemize}
  \tightlist
  \item
    18 gènes plastidiaux ont des introns (conservé chez les plantes à
    flaur)
  \item
    15 n’en ont qu’1, 3 en ont 2 = 21 introns/génome
  \end{itemize}
\end{itemize}

\hypertarget{quelles-sont-les-fonctions-des-guxe8nes-plastidiaux}{%
\subparagraph{Quelles sont les fonctions des gènes plastidiaux
?}\label{quelles-sont-les-fonctions-des-guxe8nes-plastidiaux}}

{[}{[}Pasted image 20230202101146.png{]}{]} Chez l’ensemble des plantes
supérieures ont à plus ou moins le même nombre de gènes (101-150), mais
par rapport à ce que code le génome on va trouver 2 grandes catégories
en terme de fonction: - une partie va coder des gènes qui ont un rapport
avec \textbf{l’expression} de ce génome

rRNA permettent de composer les ribosomes (permettent la traduction
d’ARNm en protéines) trouvés dans le chloroplaste les tRNA (arn de
transfert), une partie voir tous sont nécessaires pour la traduction
ribosomales proteines small (s), large (l)

\begin{itemize}
\tightlist
\item
  une partie est exclusivement liée à la photosynthèse
\end{itemize}

\begin{quote}
les mitochondries ont la même “histoire” que le plaste (endisymbiose),
et dans la mitochondrie a la même organisation en 2 parties : 1 partie
qui va se concentrer sur l’expression, et l’autre mobilisée pour la
chaine de transfert des électrons mitochondiraux
\end{quote}

Epigagus (colonne de droite), une plante épiphyte =\textgreater{} pas de
gènes associés à la trancription ni à la photosyntèse Illustration du
fonctionnement particulier du génome plastidial, en tt cas différentié
du génome nucléaire {[}{[}Pasted image 20230202102656.png{]}{]}
Ribosomes tjrs composés de 2 sous unités, de 2 ARN et des protéines

Chez procaryotes, ribosome 70S, sous unité font 30S (16ARN et 21
protéines) et 50S (23 et 5 ARN, et 31 protéines). Quand on regarde pour
les ribosomes du génome nucléaire, on voit des ribosomes typiquement
eucaryotiques (40S et 60S; ARN différents en tailles et nombre
différents) Si je compare tout ca avec les ribosomes des plastes, on a
une organisation similaire aux procaryotes :70S, même poids de sous
unité, ARN ribosomiques très distincts de ceux des ribosomes
cytoplasmiques (de type eucaryotiques), protéines nb similaire à
procaryotes {[}{[}Pasted image 20230202103954.png{]}{]} Sur le détail de
la membrane on voit un zoom du dessin de gauche. Il faut parfois plus
d’une dizaine de protéine différentes, une partie d’entre elle sont
codées par le génome plastidial, et une autre par le génome nucléaire.
Ce qui veut dire que pour que ces systèmes fonctionnent il faut
absolument que des gènes nucléaires soit exprimés dans le noyaux,
traduit dans le cytosome puis amené dans le plaste. {[}{[}Pasted image
20230202104438.png{]}{]} CP = chloro protéines : permettent de fixer la
chlorophylle. Ou sont les gènes correspondant? Concernant le
phytosystème II: hydrophobic subunit (codées par génome plastidial), et
hydrophylic subunit (codées par génome nucléaire). Un standard du
fonctionnement du chloroplaste : les gènes codant les protéines D1 et D2
ont un turnover très important, ces 2 protéines ont une régulation qui
leur est propre et qui est particulièrement controlée. psb{[} {]} pcq
photosystème II, donc II=b {[}{[}Pasted image 20230202105114.png{]}{]}
Majorité des gènes sur le tableau sont des gènes nucléaires (ceux
entourés de vert son chloroplastiques) {[}{[}Pasted image
20230202105216.png{]}{]} LHCa ou LHCb selon PSII ou PSI Ces gènes
servent de modèle pour check les interactions entre le noyau et le
chloroplaste. Ces gènes sont particulièrement représentatif de
l’évolution du plaste en chloroplaste. Quand on veut suivre le processus
de différentiation du plaste en chloroplaste, on va regarder des
marqueurs, et les gènes LHC sont étudiés systématiquement {[}{[}Pasted
image 20230202105852.png{]}{]} {[}{[}Pasted image
20230202110100.png{]}{]} {[}{[}Pasted image 20230202110113.png{]}{]} En
jaune tt les polypeptides codés par le génome nucléaire, et en vert ceux
codés par le génome plastidial. \#\#\#\#\# Les plastes ne sont donc pas
autonomes {[}{[}Pasted image 20230202110256.png{]}{]} En violet : ce qui
correspond à la transcription/traduction En vert: lié à photosynthèse En
rouge:génome nucléaire

{[}{[}Pasted image 20230202110810.png{]}{]} Protéines ammenées
majoritairement par le système toc-tic : toc passe la protéine à travers
la première membrane plastidiale (externe), tic pour membrane interne.

\hypertarget{ii.-expression-du-guxe9nome-plastidial-muxe9canismes-et-principales-ruxe9gulations}{%
\subsubsection{II. Expression du génome plastidial : mécanismes et
principales
régulations}\label{ii.-expression-du-guxe9nome-plastidial-muxe9canismes-et-principales-ruxe9gulations}}

\hypertarget{iii.-interactions-fonctionnelles-entre-le-guxe9nome-plastidial-et-le-guxe9nome-nucluxe9aire.}{%
\subsubsection{III. Interactions fonctionnelles entre le génome
plastidial et le génome
nucléaire.}\label{iii.-interactions-fonctionnelles-entre-le-guxe9nome-plastidial-et-le-guxe9nome-nucluxe9aire.}}

\end{document}
